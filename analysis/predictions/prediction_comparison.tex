\begin{longtable}{llllllllllllll}
\toprule
                                                                                                                                                                                                                                                                                                                                                                                                                                                                                                                                                                                                                                                                                                                                                                                                                                                                                                                                                                                                                                   input &     pmid & target\_serious & flan-t5-large\_serious & gpt-4\_serious & target\_patientsex & flan-t5-large\_patientsex & gpt-4\_patientsex &                                                                                                                                                                                                                                                    target\_drugs &                                                                                               flan-t5-large\_drugs &                                                                                                                                                                                                            gpt-4\_drugs &                                                                                                                                                 target\_reactions &                                                                                                                                                                                                                                           flan-t5-large\_reactions &                                                                                               gpt-4\_reactions \\
\midrule
\endfirsthead

\toprule
                                                                                                                                                                                                                                                                                                                                                                                                                                                                                                                                                                                                                                                                                                                                                                                                                                                                                                                                                                                                                                   input &     pmid & target\_serious & flan-t5-large\_serious & gpt-4\_serious & target\_patientsex & flan-t5-large\_patientsex & gpt-4\_patientsex &                                                                                                                                                                                                                                                    target\_drugs &                                                                                               flan-t5-large\_drugs &                                                                                                                                                                                                            gpt-4\_drugs &                                                                                                                                                 target\_reactions &                                                                                                                                                                                                                                           flan-t5-large\_reactions &                                                                                               gpt-4\_reactions \\
\midrule
\endhead
\midrule
\multicolumn{14}{r}{{Continued on next page}} \\
\midrule
\endfoot

\bottomrule
\endlastfoot
TITLE: Navigating Long-Term Care. ABSTRACT: Americans over age 65 constitute a larger percentage of the population each year: from 14% in 2010 (40 million elderly) to possibly 20% in 2030 (70 million elderly). In 2015, an estimated 66 million people provided care to the ill, disabled, and elderly in the United States. In 2000, according to the Centers for Disease Control and Prevention (CDC), 15 million Americans used some form of long-term care: adult day care, home health, nursing home, or hospice. In all, 13% of people over 85 years old, compared with 1% of those ages 65 to 74, live in nursing homes in the United States. Transitions of care, among these various levels of care, are common: Nursing home to hospital transfer, one of the best-studied transitions, occurs in more than 25% of nursing home residents per year. This article follows one patient through several levels of care. TEXT: Case: AB Mrs. AB is an 84-year-old Caucasian female with a history of hypertension, osteoporo... & 28491911 &              1 &                     1 &             1 &                 2 &                        2 &                2 & alendronate sodium, amitriptyline, ascorbic acid, celecoxib, chromic chloride\chromium, cinnamon, diltiazem, ginkgo, glucosamine, glyburide, hydroxyzine hydrochloride, metformin hydrochloride, niacin, paroxetine, pioglitazone, simvastatin, st. john^s wort & alendronate sodium, amitriptyline, celecoxib, glimepiride, hydroxyzine palmitate, niacin, paroxetine, simvastatin & diltiazem xr, simvastatin, amitriptyline, paroxetine, st. john's wort, celecoxib, metformin, alendronate, glyburide xr, pioglitazone, hydroxyzine palmoate, chromium, cinnamon, ginkgo, glucosamine, niacin, vitamin c &                                                cognitive disorder, delirium, dementia alzheimer^s type, drug interaction, fall, hip fracture, mobility decreased &                                                                                                                                                                                                                               drug interaction, memory impairment &                                                    cognitive impairment, drug-drug interactions, polypharmacy \\
TITLE: The Efficacy of Albumin Dialysis in the Reversal of Refractory Vasoplegic Shock Due to Amlodipine Toxicity. ABSTRACT: Calcium channel blockers are highly protein-bound medications frequently used in the management of hypertension. Overdose results in severe hypotension and is the fourth most common cause of toxicity-related deaths in the United States. Management is mostly supportive, with currently no standard role for targeted drug removal. The protein-bound nature of these medications presents the option of utilizing albumin dialysis for their removal and for the reversal of associated shock. We present two cases of life-threatening intentional amlodipine overdoses successfully treated with albumin dialysis. Both patients experienced profound distributive shock in the setting of preserved cardiac contractility that was refractory to maximal vasoactive agent support. After initiation of albumin dialysis, the patients showed rapid hemodynamic improvement and were able to be ... & 32695989 &              1 &                     1 &             1 &                 1 &                        2 &                1 &                                                                                                                                                                                                                                             amlodipine besylate &                                                                                   amlodipine besylate, lisinopril &                                                                                                                                                                                                 amlodipine, lisinopril &                                                                                                                                      intentional overdose, shock &                                                                                                                                                                                                                   intentional overdose, metabolic acidosis, shock &                                                            refractory vasoplegic shock, hypotension, overdose \\
TITLE: Cyanide poisoning in inhalation injuries. ABSTRACT: Cyanide gas forms during the combustion of synthetic polymers and should be considered in patients presenting with inhalation injuries. A persistently high lactate following adequate resuscitation may be an indicator of cyanide exposure. As cyanide poisoning can be rapidly fatal, prompt recognition and treatment of this condition is vital. TEXT: A 78‐year‐old man was admitted to a National Burns Unit following a 22% total body surface area flame burn and inhalation injury. This occurred following an explosion while lighting a gas fire in his outhouse. Despite adequate fluid resuscitation and good baseline renal function, a severe increased anion gap metabolic acidosis, with an associated elevated lactate (2.26 mmol/L), persisted. Cyanide poisoning was suspected, and hydroxocobalamin was administered. Following administration, his urine rapidly turned a characteristic red‐wine color (Figure 1). Cyanide is a mitochondrial toxi... & 33363981 &              2 &                     1 &             1 &                 1 &                        1 &                1 &                                                                                                                                                                                                                                                hydroxocobalamin &                                                                                                  hydroxocobalamin &                                                                                                                                                                                                       hydroxocobalamin &                                                                                                                                                      chromaturia &                                                                                                                                                                                                                                      blood chromaturia, red urine &                                                      cyanide poisoning, inhalation injury, metabolic acidosis \\
TITLE: EBV-associated lymphoid interstitial pneumonia in IBD patient: Case report and literature review. ABSTRACT: Lymphoid interstitial pneumonia (LIP) is categorized as a rare form of interstitial lung disease. Most cases are associated with autoimmune disease. A 78-year-old male with Crohn's disease, presented with progressive dyspnea and dry cough for few weeks. The pathology of transbronchial lung biopsy was compatible with LIP and positive cells on EBER in situ hybridization. Blood EBV viral load was 85,715 copies/mL, compatible with EBV-associated LIP. All immunosuppressive agents were discontinued, but unfortunately the patient died due to hospital-acquired infections. In addition, we reviewed all reported cases of EBV-associated LIP in literature. To our knowledge, we report herein the first case of EBV-associated LIP in an IBD patient. We postulate that LIP was the consequence from EBV reactivation, probably due to immunosuppressive agents and/or IBD itself. The physician ... & 32373453 &              1 &                     1 &             1 &                 1 &                        1 &                1 &                                                                                                                                                                                                              azathioprine, infliximab, mesalamine, prednisolone &                                                                azathioprine, infliximab, mesalamine, prednisolone &                                                                                                                                                        azathioprine, ganciclovir, infliximab, mesalazine, prednisolone &                                                                                     epstein-barr virus infection reactivation, idiopathic interstitial pneumonia &                                                                                                                                                                                                              acute respiratory failure, interstitial lung disease &    autoimmune hemolytic anemia, cytomegalovirus colitis, lymphoid interstitial pneumonia, respiratory failure \\
TITLE: Reversible Cancer Therapeutics-related Cardiac Dysfunction Complicating Intra-cardiac Thrombi. ABSTRACT: Epirubicin-based chemotherapy carries a risk of inducing heart failure, although the frequency is rare. Bevacizumab, an anti-vascular endothelial growth factor monoclonal antibody, has recently been widely used in patients with recurrent breast cancer as a first-line chemotherapeutic agent. Heart failure or arterial thromboembolism has been reported as a rare cardiovascular complication of bevacizumab. We herein report a breast cancer patient with reversible cancer therapeutics-related cardiac dysfunction associated with bevacizumab and epirubicin complicating intracardiac thrombi in the left atrium and left ventricle. This case underscores the importance of tailored medical planning according to the individual status in patients receiving anti-cancer therapies. TEXT: Introduction Anthracycline, including epirubicin-based chemotherapy, improves the survival of breast cance... & 32493855 &              1 &                     1 &             1 &                 2 &                        2 &                2 &                                                                                                                                                                                                           bevacizumab, cyclophosphamide, epirubicin, paclitaxel &                                                             bevacizumab, cyclophosphamide, epirubicin, paclitaxel &                                                                                                                                                                                                epirubicin, bevacizumab &                                                                              bundle branch block right, cardiac failure, intracardiac thrombus, pleural effusion & cardiac failure, cardiac thrombosis, cardiomegaly, bundle branch block right, dyspnoea exertional, left ventricular hypertrophy, left atrial thrombosis, left ventricular dysfunction, sinus tachycardia, sinus thrombosis, sinus thorax, ventricular hypertrophy &                          cancer therapeutics-related cardiac dysfunction, heart failure, intracardiac thrombi \\
TITLE: Ceftaroline-Associated Neutropenia: Case Series and Literature Review of Incidence, Risk Factors, and Outcomes. ABSTRACT: Ceftaroline is increasingly prescribed for "off-label" indications involving longer durations and higher doses. There have been postmarketing case reports of neutropenia among patients who have received extended durations of ceftaroline, but limited published data currently exist on its incidence and risk factors. We review a total of 37 published cases of ceftaroline-associated neutropenia including cases (n = 4) identified in our health care system. The median time from ceftaroline initiation to development of neutropenia (range) was 25 (8-125) days, with a median duration of neutropenia (range) of 4 (1-16) days. Agranulocytosis (absolute neutrophil count [ANC] nadir < 100 cells/mm3) developed in 49% of cases (n = 18), and there was an ANC nadir of 0 in 27% (n = 10). The overall incidence of neutropenia among cases receiving ceftaroline for ≥7-14 days (r... & 31123688 &              1 &                     1 &             1 &                 1 &                        1 &                1 &                                                                                                                                                                                              ceftaroline fosamil, daptomycin, famotidine, linezolid, vancomycin &                                                                                         ceftaroline hydrochloride &                                                                                                                                                                                                            ceftaroline &                                                                                                                          eosinophilia, neutropenia, pancytopenia &                                                                                                                                                                                                                                                       neutropenia &                                                                                                   neutropenia \\
TITLE: A case of bilateral human herpes virus 6 panuveitis with genomic viral DNA integration. ABSTRACT: BACKGROUND We report a rare case of bilateral panuveitis from human herpes virus 6 (HHV-6) with genomic viral DNA integration in an immunocompromised man.  RESULTS A 59-year-old man with history of multiple myeloma presented with altered mental status, bilateral eye redness, and blurry vision. Examination revealed bilateral diffuse keratic precipitates, 4+ anterior chamber cell, hypopyon, vitritis, and intraretinal hemorrhages. Intraocular fluid testing by polymerase chain reaction (PCR) was positive for HHV-6. The patient was successfully treated with intravitreal foscarnet and intravenous ganciclovir and foscarnet. Despite clinical improvement, his serum HHV-6 levels remained high, and it was concluded that he had HHV-6 chromosomal integration.  CONCLUSIONS HHV-6 should be considered in the differential for infectious uveitis in immunocompromised hosts who may otherwise have a ... & 24995045 &              1 &                     1 &             1 &                 1 &                        1 &                1 &                                                                                                                                                                                                          ceftazidime, foscarnet sodium, ganciclovir, vancomycin &                                                                   ceftazidime, foscarnet, ganciclovir, vancomycin &                                                                                                                                                                                                 foscarnet, ganciclovir &                                                                                                     hypersensitivity vasculitis, off label use, renal impairment &                                                                                                                                                                                                      leukocytoclastic vasculitis, off label use, renal impairment &                                       encephalitis, leukocytoclastic vasculitis, panuveitis, renal impairment \\
TITLE: Chromosomal rearrangement involving 11q23 locus in chronic myelogenous leukemia: a rare phenomenon frequently associated with disease progression and poor prognosis. ABSTRACT: BACKGROUND Progression of chronic myelogenous leukemia (CML) is frequently accompanied by cytogenetic evolution, commonly unbalanced chromosomal changes, such as an extra copy of Philadelphia chromosome (Ph), +8, and i(17)(q10). Balanced chromosomal translocations typically found in de novo acute myeloid leukemia occur occasionally in CML, such as inv(3)/t(3;3), t(8;21), t(15;17), and inv(16). Translocations involving the 11q23, a relatively common genetic abnormality in acute leukemia, have been seldom reported in CML. In this study, we explored the prevalence and prognostic role of 11q23 in CML.  METHODS We searched our pathology archives for CML cases diagnosed in our institution from 1998 to present. Cases with 11q23 rearrangements were retrieved. The corresponding clinicopathological data were revi... & 25888368 &              1 &                     1 &             1 &                 1 &                        1 &                1 &                                                                                                                                                                                                                       bosutinib, dasatinib, imatinib, nilotinib &                                                                                             hydroxyurea, imatinib &                                                                                                                                                                                         imatinib, dasatinib, nilotinib &                                                         blast cell count increased, chronic myeloid leukaemia transformation, drug ineffective, thrombocytopenia &                                                                                                                                                                                                                                             blast stage leukaemia &                                                         clonal evolution, disease progression, poor prognosis \\
TITLE: Pharmacokinetics and safety of panitumumab in a patient with chronic kidney disease. ABSTRACT: Data on panitumumab dosing in cancer patients with renal insufficiency are lacking. Here, we report a 63-year-old metastatic colorectal cancer patient with chronic kidney injury with a glomerular filtration rate of approximately 11 mL/min.  Pharmacokinetic parameters, including dose-normalized area under the curve, clearance and elimination half-life (T 1/2) after the 11th and 12th infusions were estimated using trapezoidal non-compartmental methods. Data were compared to previous reported pharmacokinetic data from studies in patients with normal renal function.  The results show that the pharmacokinetic data in this patient with kidney failure are comparable to those in patients with adequate renal function. Moreover the treatment was well tolerated in this patient.  This study suggests that panitumumab can be safely used in cancer patients with renal impairment without dose adjust... & 29170802 &              1 &                     1 &             1 &                 1 &                        1 &                1 &                                                                                                                                                                                                                           fluorouracil, folic acid, oxaliplatin &                                                                fluorouracil, leucovorin, oxaliplatin, panitumumab &                                                                                                                                                                                                            panitumumab &                                                                                                           product use in unapproved indication, renal impairment &                                                                                                                                                                                                                              electrolyte imbalance, skin toxicity &                                                                                                 skin toxicity \\
TITLE: Cardiac safety results from a phase II, open-label, multicenter, pilot study of two docetaxel-based regimens plus bevacizumab for the adjuvant treatment of subjects with node-positive or high-risk node-negative breast cancer. ABSTRACT: OBJECTIVE Adding antiangiogenic therapy to standard chemotherapy has improved response rates and progression-free survival in metastatic breast cancer (BC) patients. This phase II study evaluated cardiac safety of bevacizumab with/without trastuzumab with two docetaxel-based regimens in early BC.  METHODS 127 women with non-metastatic node-positive or high-risk node-negative BC were enrolled. Women with human epidermal growth factor receptor 2 (HER2)-negative BC (n = 93) received docetaxel/doxorubicin/cyclophosphamide (TAC) + bevacizumab, while women with HER2-positive disease (n = 34) received docetaxel/carboplatin/trastuzumab (TCH) + bevacizumab, every 3 weeks for six cycles. Maintenance therapy with bevacizumab alone or bevacizumab plus tras... & 24860718 &              1 &                     1 &             1 &                 2 &                        2 &                2 &                                                                                                                                                                                             bevacizumab, cyclophosphamide, docetaxel, doxorubicin hydrochloride &                                                     bevacizumab, carboplatin, docetaxel, doxorubicin, trastuzumab &                                                                                                                                        bevacizumab, trastuzumab, docetaxel, doxorubicin, cyclophosphamide, carboplatin &                                                                                                                                            clostridial infection &                                                                                                                                                                                                                                        cardiac failure congestive &                                        congestive heart failure, cardiomyopathy, left ventricular dysfunction \\
TITLE: Breast Cancer Metastasis Masquerading as Primary Colon and Gastric Cancer: A Case Report. ABSTRACT: <strong>BACKGROUND</strong> Breast cancer is the most common malignancy in women worldwide. Despite treatment, recurrence and metastasis are common. Lobular breast cancer most commonly metastasizes to the lungs, liver, lymph nodes, and sites in the brain. Metastasis to the gastrointestinal tract is rare, with few cases reported to date. <strong>CASE REPORT</strong> This report describes a patient with late colon and gastric metastases from lobular breast cancer mimicking primary colon and gastric cancers. <strong>CONCLUSIONS</strong> Immunohistochemical methods can help differentiate metastatic breast disease to the gastrointestinal tract from primary gastrointestinal malignancy. TEXT: Background The objective herein is to describe a case of a woman with late gastrointestinal (GI) metastases from breast cancer, that mimicked primary colon and gastric cancer based on clinical pr... & 31927561 &              1 &                     1 &             1 &                 2 &                        2 &                2 &                                                                                                                                                                                                                                         everolimus, fulvestrant &                            capecitabine, everolimus, exemestane, paclitaxel, gemcitabinegemcitabine hydrochloride &                                                                                                        anastrozole, letrozole, exemestane, fulvestrant, palbociclib, everolimus, capecitabine, paclitaxel, gemcitabine &                                                                              breast cancer metastatic, malignant neoplasm progression, second primary malignancy &                                                                                                                                                                                                                 multiple organ dysfunction syndrome, septic shock &                       bone metastases, colon metastases, gastric metastases, septic shock, multiorgan failure \\
TITLE: Beware of Reversal of an Anticoagulated Patient with Factor IX in the Emergency Department: Case Report of a Medical-Legal Misadventure. ABSTRACT: In this article we present a case of a patient who received reversal of anticoagulation therapy with factor IX in violation of hospital guidelines. As a direct result, myocardial infarction and ischemic stroke occurred, leaving the patient neurologically debilitated. Factor IX is indicated in the setting of warfarin-induced, life-threatening bleeding. The patient's care was provided by an intern with attending physician supervision. Delayed charting and questionable shared decision-making were present in the care. We discuss usage of factor IX, liability for supervision of physicians in training, and factors that can lead to plaintiff awards. TEXT: CASE PRESENTATION A 54-year-old woman presented to her primary care physician complaining of epistaxis, hematochezia, headache, and a seizure. She had a prior history of seizures. The pa... & 32064415 &              1 &                     1 &             1 &                 2 &                        2 &                2 &                                                                                                                                                                                                                                               factor ix complex &                                                                                               factor ix, warfarin &                                                                                                                                                                                            factor ix complex, warfarin &                                                            acute myocardial infarction, cardiac arrest, electrocardiogram st segment elevation, ischaemic stroke &                                                                                                                                                                                                                     acute myocardial infarction, ischaemic stroke &                                                                        myocardial infarction, ischemic stroke \\
TITLE: No Stones, Some Groans, and Psychiatric Overtones with "Non-specific" Splenomegaly. ABSTRACT: Hypercalcemia is a potentially life-threatening electrolyte imbalance that is commonly caused by hyperparathyroidism, supplement or medication use, and/or malignancy. Splenomegaly is commonly a non-specific finding, but in the setting of hypercalcemia, may provide diagnostic insight into the underlying pathology and warrant further evaluation. A 70-year-old man presented from his outpatient provider with serum calcium > 15 mg/dL with complaints of one-month fatigue, weakness, poor oral intake, 10 lbs. unintentional weight loss, and periodic confusion noted by his wife. He received an extensive inpatient workup which was non-diagnostic. Splenomegaly was observed on radiographic imaging and reported as "nonspecific". Following discharge, denosumab was required to manage the hypercalcemia. Eventually, a diagnosis of primary splenic lymphoma was made months later. Laparoscopic splenectom... & 31312564 &              1 &                     1 &             1 &                 1 &                        1 &                1 &                                                                                                                                                                                    amlodipine besylate, calcitonin, denosumab, furosemide, pamidronate disodium &                                                                                               amlodipine besylate &                                                                                                                                           denosumab, rituximab, cyclophosphamide, doxorubicin, vincristine, prednisone &                                                                                                                                    diffuse large b-cell lymphoma &                                                                                                                                                                                                acute kidney injury, hypercalcaemia, splenic large b-cell lymphoma &                                                            hypercalcemia, splenomegaly, large b-cell lymphoma \\
TITLE: Carcinoid Crisis-Induced Acute Systolic Heart Failure. ABSTRACT: Carcinoid crisis is a life-threatening manifestation of carcinoid syndrome characterized by profound autonomic instability in the setting of catecholamine release from stress, tumor manipulation, or anesthesia. Here, we present an unusual case of carcinoid crisis leading to acute systolic heart failure requiring mechanical circulatory support. (Level of Difficulty: Intermediate.). TEXT: History of Present Illness A 59-year-old woman with a stage IV, World Health Organization grade 2 ileal neuroendocrine tumor was referred for transcatheter arterial chemoembolization (TACE) of extensive bilobar hepatic metastases (Figure 1A). Two years prior, she underwent ileocolectomy and began octreotide depot injections (escalated to 60 mg/month) to manage diarrhea and flushing from carcinoid syndrome. Transthoracic echocardiography performed 1 month prior to TACE revealed normal biventricular size and function.Learning Objec... & 34317110 &              1 &                     1 &             1 &                 2 &                        2 &                2 &                                                                                                                                                      dobutamine hydrochloride, doxorubicin, milrinone, mitomycin, octreotide, sodium nitroprusside, vasopressin &                                                         dobutaminedobutamine hydrochloride, mitomycin, octreotide &                                                                                                                                                                                     doxorubicin, mitomycin, octreotide &                                                                                                                                                 drug ineffective &                                                                                                                                                           cardiogenic shock, drug interaction, drug level increased, pulmonary oedema, respiratory failure, shock &                                            acute systolic heart failure, carcinoid crisis, coronary vasospasm \\
TITLE: Drug reaction with eosinophilia and systemic symptoms (DRESS) with severe and atypical lung involvement. ABSTRACT: Drug reaction with eosinophilia and systemic symptoms is a rare and potentially fatal drug hypersensitivity reaction. Reactions include skin eruption, fever, hematologic abnormalities (eosinophilia or atypical lymphocytosis), enlarged lymph nodes, and/or organic involvement. The liver is the most commonly compromised organ. We present a case of drug reaction with eosinophilia and systemic symptoms associated with Naproxen intake in a young female patient with severe lung involvement. The patient's chest tomography highlights the presence of adenomegalies, pericardial and pleural effusion, peribroncovascular consolidations, and centrilobular nodules. After reviewing the literature few similar cases were found. The main radiological alterations in those cases included interstitial opacities attributed to pneumonitis. Therefore, this case study is considered an unus... & 32944113 &              1 &                     1 &             1 &                 2 &                        2 &                2 &                                                                                                                                                                                                                                                        naproxen &                                                                                                          naproxen &                                                                                                                                                                                                               naproxen & acute hepatic failure, drug reaction with eosinophilia and systemic symptoms, eosinophilic pneumonia, lung consolidation, pericardial effusion, pleural effusion &                                                                                                                                                                                                                     acute hepatic failure, eosinophilic pneumonia &            drug reaction with eosinophilia and systemic symptoms, acute liver failure, eosinophilic pneumonia \\
TITLE: Sirolimus for Kaposiform Hemangioendothelioma and Kasabach-Merritt Phenomenon in a Neonate. ABSTRACT: We present a case of a neonate born with kaposiform hemangioendothelioma (KHE), complicated by Kasabach-Merritt phenomenon (KMP) and other serious conditions, who was successfully treated with sirolimus. In addition to complications from thrombocytopenia and fluid overload, during the course of therapy, our patient experienced supratherapeutic drug levels at the commonly accepted starting dose of sirolimus. Patients with KHE and KMP should be closely monitored for potential complications of both the initial disease and unexpected side effects of treatments. TEXT: Kaposiform hemangioendothelioma (KHE) is a rare infiltrative vascular tumor typically diagnosed during infancy. Over 70% of patients with KHE will develop Kasabach-Merritt phenomenon (KMP), a potentially life-threatening consumptive coagulopathy. We present the case of an infant with KHE and KMP successfully treated ... & 33214934 &              2 &                     1 &             1 &                 2 &                        1 &                1 &                                                                                                                                                                                                            betamethasone acetate\betamethasone sodium phosphate &                                                                                     methylprednisolone, sirolimus &                                                                                                                                                                                                              sirolimus &                                                                                                                               maternal exposure during pregnancy &                                                                         anasarca, acidosis, drug level above therapeutic, fluid overload, hypoventricle dilatation, mitral valve incompetence, occipital haematoma, periventricular enlargement, thrombocytopenia &                                 anasarca, acidosis, intraventricular hemorrhage, supratherapeutic drug levels \\
TITLE: Unexpected Acute Necrotizing Ulcerative Gingivitis in a Well-controlled HIV-infected Case. ABSTRACT: We herein report the case of a 41-year-old Japanese man with well-controlled HIV who presented with diagnostically difficult acute necrotizing ulcerative gingivitis (ANUG). After diet-induced weight loss, he developed oral pain and disturbance of mouth opening, and was admitted to our hospital. Based on preconceptions of HIV-associated diseases, fluconazole was initiated for candidiasis. However, no improvement was seen and ANUG was finally diagnosed. This case suggests that physicians should consider ANUG in HIV-infected individuals when several risk factors are present, even if CD4+ T-lymphocyte counts have remained stable owing to long-term anti-retroviral therapy. TEXT: Introduction Acute necrotizing ulcerative gingivitis (ANUG) is a periodontal disease, the main features of which are infection and inflammation of the periodontal tissues. ANUG is characterized by marginal ... & 28781315 &              1 &                     1 &             1 &                 1 &                        1 &                1 &                                                                                                                                                                                                                abacavir sulfate\lamivudine, lopinavir\ritonavir &                                                                          abacavir, lamivudine, lopinavirritonavir &                                                                                                                                                     abacavir/lamivudine, lopinavir/ritonavir, fluconazole, clindamycin &                                                                                                            breath odour, gingivitis ulcerative, oral candidiasis &                                                                                                                                                                                                                           acute necrotising ulcerative gingivitis & acute necrotizing ulcerative gingivitis, oral pain, fever, general debilitation, disturbance of mouth opening \\
TITLE: Cholestasis and disseminated histoplasmosis in a psoriatic patient on infliximab: case report and review of literature. ABSTRACT: BACKGROUND Histoplasma capsulatum is the most common endemic mycosis in the United States and frequently presents as an opportunistic infection in immunocompromised hosts. Though liver involvement is common in disseminated histoplasmosis, primary gastrointestinal histoplasmosis of the liver in absence of lung involvement is rare. Similarly, cholestatic granulomatous hepatitis in liver histoplasmosis is rarely seen.  METHODS We present a rare case of primary gastrointestinal histoplasmosis manifesting with acute granulomatous hepatitis and cholestasis in a 48-year-old female with psoriatic arthritis, receiving methotrexate and infliximab. The epidemiology, risk factors, clinical presentation, diagnosis, and treatment of histoplasmosis is discussed. Furthermore, we review the published cases of biopsy-proven disseminated histoplasmosis with cholestat... & 32384881 &              1 &                     1 &             1 &                 2 &                        2 &                2 &                                                                                                                                                                                                                                        infliximab, methotrexate &                                                                                          infliximab, methotrexate &                                                                                                                                                                                               infliximab, methotrexate &                                                       cardiac tamponade, cholestasis, granulomatous liver disease, histoplasmosis disseminated, pleural effusion &                                                                                                                                                                                               histoplasmosis disseminated, jaundice cholestatic, pleural effusion &                                             cholestasis, disseminated histoplasmosis, granulomatous hepatitis \\
TITLE: Severe hyperkalemia following adrenalectomy for aldosteronoma: prediction, pathogenesis and approach to clinical management- a case series. ABSTRACT: BACKGROUND As the field of Primary Aldosteronism (PA) becomes ever expanded, diagnosis of PA is increasingly diagnosed by endocrinologists. With increased PA screening, many of the cases are now found in patients with complex co-morbidities in addition to their hypertension. Post adrenalectomy renal impairment with hyperkalemia is now increasingly seen in these complex patients, as evidenced by the numerous reports on this issue that have appeared within the past 3 years. We present a small case series to illustrate the breadth of the problem, along with a discussion about how such CKD/hyperkalemic events may be predicted.  METHODS We present three cases of primary aldosteronism with long standing hypertension (more than 10 years) hypokalemia (2.0-3.0 mmol/l). Serum aldosterone was high with low renin activity leading to high al... & 27460219 &              1 &                     1 &             1 &                 2 &                        2 &                2 &                                                                                                                                                                                                                                                      furosemide &                                                                              amlodipine besylate, fludrocortisone &                                                                                                                                                                        fludrocortisone, furosemide, sodium bicarbonate &                                                                                                           blood creatine increased, hyperkalaemia, hyponatraemia &                                                                                                                                                                                                                         blood creatinine increased, hyperkalaemia &                                                                                hyperkalemia, renal impairment \\
TITLE: Hypertrophic pyloric stenosis following persistent pulmonary hypertension of the newborn: a case report and literature review. ABSTRACT: Although persistent pulmonary hypertension of the newborn (PPHN) and infantile hypertrophic pyloric stenosis (HPS) are both well-known diseases that occur in early infancy, PPHN complicated by HPS is rare. As nitric oxide (NO) is an important mediator of biological functions, on both the vascular endothelium and smooth muscle cells, the decreased production of NO might play a role in the pathogenesis of both PPHN and HPS. We present the case of a neonate who developed HPS following PPHN, including a detailed review on research published to date, and we discuss the pathogenesis of PPHN and HPS.  A female neonate born at 38 weeks of gestation, weighing 3140 g, developed PPHN due to meconium aspiration syndrome. Intensive treatment with high frequency oscillations and inhaled NO were initiated, and sildenafil and bosentan were added. She gradua... & 30176827 &              1 &                     1 &             1 &                 2 &                        2 &                2 &                                                                                                                                                                                                                              bosentan, nitric oxide, sildenafil &                                                                                    atropine, bosentan, sildenafil &                                                                                                                                                nitric oxide, sildenafil, bosentan, atropine, transdermal nitroglycerin &                                                                                                                                       pyloric stenosis, vomiting &                                                                                                                                                                                                                   drug ineffective, hypertrophic pyloric stenosis &                               persistent pulmonary hypertension of the newborn, hypertrophic pyloric stenosis \\
\end{longtable}
