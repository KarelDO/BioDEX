\begin{longtable}{@{}p{1.00\textwidth}@{}}
\toprule
\textbf{input} \tabularnewline
\midrule
\endhead
\bottomrule
\endfoot
\endlastfoot
(PMID: 28491911) TITLE: Navigating Long{-}Term Care. ABSTRACT: Americans over age 65 constitute a larger percentage of the population each year: from 14\% in 2010 (40 million elderly) to possibly 20\% in 2030 (70 million elderly). In 2015, an estimated 66 million people provided care to the ill, disabled, and elderly in the United States. In 2000, according to the Centers for Disease Control and Prevention (CDC), 15 million Americans used some form of long{-}term care: adult day care, home health, nursing home, or hospice. In all, 13\% of people over 85 years old, compared with 1\% of those ages 65 to 74, live in nursing homes in the United States. Transitions of care, among these various levels of care, are common: Nursing home to hospital transfer, one of the best{-}studied transitions, occurs in more than 25\% of nursing home residents per year. This article follows one patient through several levels of care. TEXT: Case: AB Mrs. AB is an 84{-}year{-}old Caucasian female with a history of hypertension, osteoporosis, type 2 diabetes, dyslipidemia, osteoarthritis, and persistent depression who presents to the office as a new patient with worsening ambulation: “I’m just not getting around well.” The patient lives in a small house above the family farm, on the side of a mountain. She describes her difficulty as an unsteadiness, and stiffness, in her knees and hips. She has moderate pain in her right hip and in her left knee, especially late in the day. On clinical examination, AB has reduced internal and external rotation of the hips, right side more affected than the left, with some pain to the maneuvers, and widened knees with some tenderness. A Mini{-}Mental Status Exam (MMSE) is consistent with mild cognitive impairment, with a score of 24. (Generally, scores of 27{-}30 are normal, 24{-}26 suggest mild cognitive impairment, 19{-}23 mild dementia, 10{-}18 moderate dementia, and <10 severe dementia.) She is taking 17 different medications, listed in the box below. AB has a son, Fred, who lives in the main farmhouse below her house, but AB does not get along with him well: He has a diagnosis of bipolar, and they argue frequently. Her other son, Rod, lives in Texas, and has recently been diagnosed with leukemia. Rod helps her with medical decisions—For example, he helped her pick her current Medicare part D plan. AB also has one surviving brother, 89 years old, but he is rather debilitated. He lives close by her house, but is unable to assist her; in fact, she assists him—sh... [Truncated] \\ \\ \begin{tabular}{lll} \hline & serious & patientsex \\ \hline target & 1 & 2 \\ flan-t5-large & 1 & 2 \\ gpt-4 & 1 & 2 \\ \hline & \multicolumn{2}{l}{drugs} \\ \hline target & \multicolumn{2}{p{13.2cm}}{alendronate sodium, amitriptyline, ascorbic acid, celecoxib, chromic chloride\textbackslash{}chromium, cinnamon, diltiazem, ginkgo, glucosamine, glyburide, hydroxyzine hydrochloride, metformin hydrochloride, niacin, paroxetine, pioglitazone, simvastatin, st. john\^{}s wort} \\ flan-t5-large & \multicolumn{2}{p{13.2cm}}{alendronate sodium, amitriptyline, celecoxib, glimepiride, hydroxyzine palmitate, niacin, paroxetine, simvastatin} \\ gpt-4 & \multicolumn{2}{p{13.2cm}}{diltiazem xr, simvastatin, amitriptyline, paroxetine, st. john's wort, celecoxib, metformin, alendronate, glyburide xr, pioglitazone, hydroxyzine palmoate, chromium, cinnamon, ginkgo, glucosamine, niacin, vitamin c} \\ \hline & \multicolumn{2}{l}{reactions} \\ \hline target & \multicolumn{2}{p{13.2cm}}{cognitive disorder, delirium, dementia alzheimer\^{}s type, drug interaction, fall, hip fracture, mobility decreased} \\ flan-t5-large & \multicolumn{2}{p{13.2cm}}{drug interaction, memory impairment} \\ gpt-4 & \multicolumn{2}{p{13.2cm}}{cognitive impairment, drug{-}drug interactions, polypharmacy} \\ \hline \end{tabular} \\ \\
 (PMID: 32695989) TITLE: The Efficacy of Albumin Dialysis in the Reversal of Refractory Vasoplegic Shock Due to Amlodipine Toxicity. ABSTRACT: Calcium channel blockers are highly protein{-}bound medications frequently used in the management of hypertension. Overdose results in severe hypotension and is the fourth most common cause of toxicity{-}related deaths in the United States. Management is mostly supportive, with currently no standard role for targeted drug removal. The protein{-}bound nature of these medications presents the option of utilizing albumin dialysis for their removal and for the reversal of associated shock. We present two cases of life{-}threatening intentional amlodipine overdoses successfully treated with albumin dialysis. Both patients experienced profound distributive shock in the setting of preserved cardiac contractility that was refractory to maximal vasoactive agent support. After initiation of albumin dialysis, the patients showed rapid hemodynamic improvement and were able to be weaned off vasopressor support. These cases demonstrate the safety and efficacy of albumin dialysis in the management of near{-}fatal calcium channel blocker overdoses related to amlodipine and offer an additional therapeutic option apart from conventional supportive care. Importantly, these cases were not associated with impaired cardiac contractility, thereby making venoarterial extracorporeal membrane oxygenation a less preferable option. Furthermore, this therapeutic benefit of albumin dialysis can potentially be extended to the management of toxicity related to other highly protein{-}bound drugs and toxins. TEXT: According to the National Poison Data System, calcium channel blocker (CCB) toxicity was the fourth highest cause of toxicity{-}related deaths in 2016, accounting for over 5\% of fatal exposures (1). Non{-}dihydropyridine CCB (e.g., verapamil, diltiazem) toxicity can cause negative inotropic and chronotropic effects, in particular, resulting in life{-}threatening cardiogenic shock. Typical therapies are supportive, aimed to temporize hemodynamic derangements until inherent elimination can occur. Venoarterial extracorporeal membrane oxygenation (VA{-}ECMO) is an additional therapeutic option in this context. In contrast, dihydropyridine CCB (e.g., amlodipine) toxicity is predominantly associated with systemic vasodilation and less cardiac depression, thereby resulting in distributive shock; in this setting, VA{-}ECMO has not traditionally been used given pre... [Truncated] \\ \\ \begin{tabular}{lll} \hline & serious & patientsex \\ \hline target & 1 & 1 \\ flan-t5-large & 1 & 2 \\ gpt-4 & 1 & 1 \\ \hline & \multicolumn{2}{l}{drugs} \\ \hline target & \multicolumn{2}{p{13.2cm}}{amlodipine besylate} \\ flan-t5-large & \multicolumn{2}{p{13.2cm}}{amlodipine besylate, lisinopril} \\ gpt-4 & \multicolumn{2}{p{13.2cm}}{amlodipine, lisinopril} \\ \hline & \multicolumn{2}{l}{reactions} \\ \hline target & \multicolumn{2}{p{13.2cm}}{intentional overdose, shock} \\ flan-t5-large & \multicolumn{2}{p{13.2cm}}{intentional overdose, metabolic acidosis, shock} \\ gpt-4 & \multicolumn{2}{p{13.2cm}}{refractory vasoplegic shock, hypotension, overdose} \\ \hline \end{tabular} \\ \\
 (PMID: 33363981) TITLE: Cyanide poisoning in inhalation injuries. ABSTRACT: Cyanide gas forms during the combustion of synthetic polymers and should be considered in patients presenting with inhalation injuries. A persistently high lactate following adequate resuscitation may be an indicator of cyanide exposure. As cyanide poisoning can be rapidly fatal, prompt recognition and treatment of this condition is vital. TEXT: A 78‐year‐old man was admitted to a National Burns Unit following a 22\% total body surface area flame burn and inhalation injury. This occurred following an explosion while lighting a gas fire in his outhouse. Despite adequate fluid resuscitation and good baseline renal function, a severe increased anion gap metabolic acidosis, with an associated elevated lactate (2.26~mmol/L), persisted. Cyanide poisoning was suspected, and hydroxocobalamin was administered. Following administration, his urine rapidly turned a characteristic red‐wine color (Figure~1). Cyanide is a mitochondrial toxin which preferentially binds ferric ions in cytochrome oxidase a3—inhibiting this final enzyme in the mitochondrial cytochrome complex. This causes oxidative phosphorylation to cease. Cells switch to anaerobic metabolism leading to the formation of lactic acid and a metabolic acidosis. 1 Hydroxocobalamin is a synthetic form of vitamin B12 which binds cyanide and forms the nontoxic cyanocobalamin. This is renally cleared, giving the urine a dark red color. Onset of chromaturia typically occurs within the first 2~hours following administration and can persist for up to 35~days. 2 Figure 1 Red‐wine colored urine as a result of hydroxocobalamin administration Cyanide gas forms during the combustion of synthetic polymers often found in building materials and furnishings. As cyanide gas can be rapidly fatal, a low threshold for treatment should exist in those suspected of having inhalation injuries. Within 7~hours of administration of hydroxocobalamin, the patient's acidosis had resolved and his lactate had significantly improved (1.49~mmol/L). As expected, his urine remained discolored for approximately three weeks. After a protracted hospital stay, the patient was discharged home well and has since returned to work in his family business. CONFLICT OF INTEREST None declared. AUTHOR CONTRIBUTIONS SK: drafted and reviewed the article. KC: reviewed the article. ETHICAL APPROVAL The regional Research Ethics Committee judged that this work was exempt from ethical review. ACKNOWLEDGMENT... [Truncated] \\ \\ \begin{tabular}{lll} \hline & serious & patientsex \\ \hline target & 2 & 1 \\ flan-t5-large & 1 & 1 \\ gpt-4 & 1 & 1 \\ \hline & \multicolumn{2}{l}{drugs} \\ \hline target & \multicolumn{2}{p{13.2cm}}{hydroxocobalamin} \\ flan-t5-large & \multicolumn{2}{p{13.2cm}}{hydroxocobalamin} \\ gpt-4 & \multicolumn{2}{p{13.2cm}}{hydroxocobalamin} \\ \hline & \multicolumn{2}{l}{reactions} \\ \hline target & \multicolumn{2}{p{13.2cm}}{chromaturia} \\ flan-t5-large & \multicolumn{2}{p{13.2cm}}{blood chromaturia, red urine} \\ gpt-4 & \multicolumn{2}{p{13.2cm}}{cyanide poisoning, inhalation injury, metabolic acidosis} \\ \hline \end{tabular} \\ \\
 (PMID: 32373453) TITLE: EBV{-}associated lymphoid interstitial pneumonia in IBD patient: Case report and literature review. ABSTRACT: Lymphoid interstitial pneumonia (LIP) is categorized as a rare form of interstitial lung disease. Most cases are associated with autoimmune disease. A 78{-}year{-}old male with Crohn's disease, presented with progressive dyspnea and dry cough for few weeks. The pathology of transbronchial lung biopsy was compatible with LIP and positive cells on EBER in situ hybridization. Blood EBV viral load was 85,715 copies/mL, compatible with EBV{-}associated LIP. All immunosuppressive agents were discontinued, but unfortunately the patient died due to hospital{-}acquired infections. In addition, we reviewed all reported cases of EBV{-}associated LIP in literature. To our knowledge, we report herein the first case of EBV{-}associated LIP in an IBD patient. We postulate that LIP was the consequence from EBV reactivation, probably due to immunosuppressive agents and/or IBD itself. The physician should aware of this disease when taking care of immunosuppressive patients who present with acute interstitial pneumonitis. TEXT: 1 Introduction Lymphoid interstitial pneumonia (LIP) is categorized as a rare form of interstitial lung disease according to the classification of American Thoracic Society/European Respiratory Society {[}1{]}. The definite diagnosis requires both imagings and pathology. Chest computed tomogram reveals the presence of ground glass attenuation, centrilobular and subpleural nodules, and thickening of bronchovascular bundles. The pathologic are characterized by the presence of dense polyclonal interstitial lymphocytic infiltrates with widening interlobular and alveolar septa {[}2,3{]}. Most cases are associated with autoimmune disease or lymphoproliferative disorder {[}4{]}. EBV, a double{-}stranded DNA virus, belongs to the Herpesviridae family {[}5{]}. EBV is able to cause latent infection, and reactivation occurs when infected individuals develop immunosuppressive state. Primary EBV infection causes infectious mononucleosis syndrome, and chronic infection/reactivation can cause lymphoma, and lymphoproliferative disorder including post transplant lymphoproliferative disease (LPD) {[}6{]}. In latent phase of infection, viral protein has the ability to transform mature B lymphocyte, resulting in uncontrolled its proliferation, as LPD {[}7{]}. Inflammatory bowel diseases (IBDs), including Crohn's disease and ulcerative colitis, have been re... [Truncated] \\ \\ \begin{tabular}{lll} \hline & serious & patientsex \\ \hline target & 1 & 1 \\ flan-t5-large & 1 & 1 \\ gpt-4 & 1 & 1 \\ \hline & \multicolumn{2}{l}{drugs} \\ \hline target & \multicolumn{2}{p{13.2cm}}{azathioprine, infliximab, mesalamine, prednisolone} \\ flan-t5-large & \multicolumn{2}{p{13.2cm}}{azathioprine, infliximab, mesalamine, prednisolone} \\ gpt-4 & \multicolumn{2}{p{13.2cm}}{azathioprine, ganciclovir, infliximab, mesalazine, prednisolone} \\ \hline & \multicolumn{2}{l}{reactions} \\ \hline target & \multicolumn{2}{p{13.2cm}}{epstein{-}barr virus infection reactivation, idiopathic interstitial pneumonia} \\ flan-t5-large & \multicolumn{2}{p{13.2cm}}{acute respiratory failure, interstitial lung disease} \\ gpt-4 & \multicolumn{2}{p{13.2cm}}{autoimmune hemolytic anemia, cytomegalovirus colitis, lymphoid interstitial pneumonia, respiratory failure} \\ \hline \end{tabular} \\ \\
 (PMID: 32493855) TITLE: Reversible Cancer Therapeutics{-}related Cardiac Dysfunction Complicating Intra{-}cardiac Thrombi. ABSTRACT: Epirubicin{-}based chemotherapy carries a risk of inducing heart failure, although the frequency is rare. Bevacizumab, an anti{-}vascular endothelial growth factor monoclonal antibody, has recently been widely used in patients with recurrent breast cancer as a first{-}line chemotherapeutic agent. Heart failure or arterial thromboembolism has been reported as a rare cardiovascular complication of bevacizumab. We herein report a breast cancer patient with reversible cancer therapeutics{-}related cardiac dysfunction associated with bevacizumab and epirubicin complicating intracardiac thrombi in the left atrium and left ventricle. This case underscores the importance of tailored medical planning according to the individual status in patients receiving anti{-}cancer therapies. TEXT: Introduction Anthracycline, including epirubicin{-}based chemotherapy, improves the survival of breast cancer patients but is associated with an increased risk of heart failure (1). In recent years, systemic therapy targeting vascular endothelial growth factor (VEGF) and its receptors has proven to be a successful strategy in patients with cancer. Bevacizumab is a widely used anti{-}VEGF monoclonal antibody targeting the VEGF ligand. Although it has been shown to improve clinical outcomes in several malignancies including advanced breast cancer (2), its use has been associated with many cardiovascular events (3{-}5). We herein report a breast cancer patient with reversible cancer therapeutics{-}related cardiac dysfunction associated with bevacizumab along with epirubicin complicated by intracardiac thrombi in the left atrium and left ventricle. Case Report A 65{-}year{-}old woman with a history of postoperative chemotherapy for right breast cancer was referred to our department due to congestive heart failure. The breast cancer had been graded as clinical stage IIa, triple{-}negative invasive ductal carcinoma {[}estrogen receptor 0\%, progressive receptor 0\%, and human epidermal growth factor receptor 2 (HER2) immunohistochemistry 0\%{]}, and the Ki{-}67{-}positive cell index was 98.6\%. She had received 4 courses of epirubicin (total dose: 327 mg/m2) and cyclophosphamide (total dose: 2,183 mg/m2) followed by paclitaxel (total dose: 727 mg/m2) and bevacizumab (total dose: 546 mg/m2). Nine months after the end of epirubicin administration and three months after the end... [Truncated] \\ \\ \begin{tabular}{lll} \hline & serious & patientsex \\ \hline target & 1 & 2 \\ flan-t5-large & 1 & 2 \\ gpt-4 & 1 & 2 \\ \hline & \multicolumn{2}{l}{drugs} \\ \hline target & \multicolumn{2}{p{13.2cm}}{bevacizumab, cyclophosphamide, epirubicin, paclitaxel} \\ flan-t5-large & \multicolumn{2}{p{13.2cm}}{bevacizumab, cyclophosphamide, epirubicin, paclitaxel} \\ gpt-4 & \multicolumn{2}{p{13.2cm}}{epirubicin, bevacizumab} \\ \hline & \multicolumn{2}{l}{reactions} \\ \hline target & \multicolumn{2}{p{13.2cm}}{bundle branch block right, cardiac failure, intracardiac thrombus, pleural effusion} \\ flan-t5-large & \multicolumn{2}{p{13.2cm}}{cardiac failure, cardiac thrombosis, cardiomegaly, bundle branch block right, dyspnoea exertional, left ventricular hypertrophy, left atrial thrombosis, left ventricular dysfunction, sinus tachycardia, sinus thrombosis, sinus thorax, ventricular hypertrophy} \\ gpt-4 & \multicolumn{2}{p{13.2cm}}{cancer therapeutics{-}related cardiac dysfunction, heart failure, intracardiac thrombi} \\ \hline \end{tabular} \\ \\
 (PMID: 31123688) TITLE: Ceftaroline{-}Associated Neutropenia: Case Series and Literature Review of Incidence, Risk Factors, and Outcomes. ABSTRACT: Ceftaroline is increasingly prescribed for "off{-}label" indications involving longer durations and higher doses. There have been postmarketing case reports of neutropenia among patients who have received extended durations of ceftaroline, but limited published data currently exist on its incidence and risk factors. We review a total of 37 published cases of ceftaroline{-}associated neutropenia including cases (n = 4) identified in our health care system. The median time from ceftaroline initiation to development of neutropenia (range) was 25 (8{-}125) days, with a median duration of neutropenia (range) of 4 (1{-}16) days. Agranulocytosis (absolute neutrophil count {[}ANC{]} nadir < 100 cells/mm3) developed in 49\% of cases (n = 18), and there was an ANC nadir of 0 in 27\% (n = 10). The overall incidence of neutropenia among cases receiving ceftaroline for ≥7{-}14 days (range) was 12\% (7\%{-}18\% per individual study), higher than for comparator antibiotics in the literature. Risk factors for ceftaroline{-}associated neutropenia varied among studies and remain poorly defined. TEXT: The development of novel antibiotics is important in addressing the growing rates of antibiotic resistance. For instance, Staphylococcus aureus remains a leading cause of bacteremia and endocarditis, with an increasing preponderance due to methicillin{-}resistant S. aureus (MRSA) strains {[}1, 2{]}. Given the limitations of the currently available antibiotics (eg, vancomycin) for treating MRSA infections, including drug intolerance, adverse events, and/or clinical failure {[}3, 4{]}, new antibiotics with anti{-}MRSA activity have been recently developed. Ceftaroline (Teflaro®) gained Food and Drug Administration (FDA) approval in 2010 and is the first licensed cephalosporin that includes coverage against MRSA. Studies leading to its approval include 2 clinical trials on community{-}acquired bacterial pneumonia (CABP; FOCUS 1 and FOCUS 2) {[}5, 6{]} and 2 additional studies on acute bacterial skin and skin structure infections (ABSSSIs; CANVAS 1 and CANVAS 2) {[}7, 8{]}. These 4 studies evaluated a total of 1307 subjects, with the most common adverse events among those receiving ceftaroline being diarrhea, nausea, and rash; no patient developed neutropenia. All studies utilized a ceftaroline dosage of 600 mg intravenously (IV) every 12 hours for dura... [Truncated] \\ \\ \begin{tabular}{lll} \hline & serious & patientsex \\ \hline target & 1 & 1 \\ flan-t5-large & 1 & 1 \\ gpt-4 & 1 & 1 \\ \hline & \multicolumn{2}{l}{drugs} \\ \hline target & \multicolumn{2}{p{13.2cm}}{ceftaroline fosamil, daptomycin, famotidine, linezolid, vancomycin} \\ flan-t5-large & \multicolumn{2}{p{13.2cm}}{ceftaroline hydrochloride} \\ gpt-4 & \multicolumn{2}{p{13.2cm}}{ceftaroline} \\ \hline & \multicolumn{2}{l}{reactions} \\ \hline target & \multicolumn{2}{p{13.2cm}}{eosinophilia, neutropenia, pancytopenia} \\ flan-t5-large & \multicolumn{2}{p{13.2cm}}{neutropenia} \\ gpt-4 & \multicolumn{2}{p{13.2cm}}{neutropenia} \\ \hline \end{tabular} \\ \\
 (PMID: 24995045) TITLE: A case of bilateral human herpes virus 6 panuveitis with genomic viral DNA integration. ABSTRACT: BACKGROUND We report a rare case of bilateral panuveitis from human herpes virus 6 (HHV{-}6) with genomic viral DNA integration in an immunocompromised man. RESULTS A 59{-}year{-}old man with history of multiple myeloma presented with altered mental status, bilateral eye redness, and blurry vision. Examination revealed bilateral diffuse keratic precipitates, 4+ anterior chamber cell, hypopyon, vitritis, and intraretinal hemorrhages. Intraocular fluid testing by polymerase chain reaction (PCR) was positive for HHV{-}6. The patient was successfully treated with intravitreal foscarnet and intravenous ganciclovir and foscarnet. Despite clinical improvement, his serum HHV{-}6 levels remained high, and it was concluded that he had HHV{-}6 chromosomal integration. CONCLUSIONS HHV{-}6 should be considered in the differential for infectious uveitis in immunocompromised hosts who may otherwise have a negative work{-}up. HHV{-}6 DNA integration may lead to difficulties in disease diagnosis and determining disease resolution. TEXT: Findings Human herpes virus{-}6 (HHV{-}6) is a ubiquitous virus that infects most children by the age of three years. While the seroprevalence in the adult population approaches 95\%, and HHV{-}6 reactivations are known to be common after organ transplantation, clinical disease is rare after the primary infection {[}1{]}. Although HHV{-}6 is closely related to cytomegalovirus (CMV), ocular disease due to HHV{-}6 has been described in very few patients {[}2{-}8{]}. We report the case of an immunocompromised man who presented with encephalitis and severe bilateral panuveitis as a result of HHV{-}6 reactivation. Integration of the viral genome into the host DNA, a unique characteristic of HHV{-}6, complicated the clinical management of our patient. Case report A 59{-}year{-}old man with a history of multiple myeloma status post allogeneic stem cell transplant was admitted to our hospital with fevers and a soft tissue infection. On the fourth day of hospitalization, he developed a headache, somnolence, bilateral eye redness, and blurred vision. On presentation, his best{-}corrected visual acuity was 20/100 in the right eye and unobtainable in the left eye due to his altered mental status. The pupils were equal bilaterally with a brisk direct response and no relative afferent pupillary defect. His intraocular pressure was 5~mmHg bilaterally... [Truncated] \\ \\ \begin{tabular}{lll} \hline & serious & patientsex \\ \hline target & 1 & 1 \\ flan-t5-large & 1 & 1 \\ gpt-4 & 1 & 1 \\ \hline & \multicolumn{2}{l}{drugs} \\ \hline target & \multicolumn{2}{p{13.2cm}}{ceftazidime, foscarnet sodium, ganciclovir, vancomycin} \\ flan-t5-large & \multicolumn{2}{p{13.2cm}}{ceftazidime, foscarnet, ganciclovir, vancomycin} \\ gpt-4 & \multicolumn{2}{p{13.2cm}}{foscarnet, ganciclovir} \\ \hline & \multicolumn{2}{l}{reactions} \\ \hline target & \multicolumn{2}{p{13.2cm}}{hypersensitivity vasculitis, off label use, renal impairment} \\ flan-t5-large & \multicolumn{2}{p{13.2cm}}{leukocytoclastic vasculitis, off label use, renal impairment} \\ gpt-4 & \multicolumn{2}{p{13.2cm}}{encephalitis, leukocytoclastic vasculitis, panuveitis, renal impairment} \\ \hline \end{tabular} \\ \\
 (PMID: 25888368) TITLE: Chromosomal rearrangement involving 11q23 locus in chronic myelogenous leukemia: a rare phenomenon frequently associated with disease progression and poor prognosis. ABSTRACT: BACKGROUND Progression of chronic myelogenous leukemia (CML) is frequently accompanied by cytogenetic evolution, commonly unbalanced chromosomal changes, such as an extra copy of Philadelphia chromosome (Ph), +8, and i(17)(q10). Balanced chromosomal translocations typically found in de novo acute myeloid leukemia occur occasionally in CML, such as inv(3)/t(3;3), t(8;21), t(15;17), and inv(16). Translocations involving the 11q23, a relatively common genetic abnormality in acute leukemia, have been seldom reported in CML. In this study, we explored the prevalence and prognostic role of 11q23 in CML. METHODS We searched our pathology archives for CML cases diagnosed in our institution from 1998 to present. Cases with 11q23 rearrangements were retrieved. The corresponding clinicopathological data were reviewed. RESULTS A total of 2,012 cases of CML with available karyotypes were identified. Ten (0.5\%) CML cases had 11q23 rearrangement in Ph{-}positive cells, including 4 cases of t(9;11), 2 cases of t(11;19), and 1 case each of t(2;11), t(4;11), t(6;11), and t(4;9;11). Eight cases (80\%) had other concurrent chromosomal abnormalities. There were 6 men and 4 women with a median age of 50 years (range, 21{-}70 years) at time of initial diagnosis of CML. 11q23 rearrangement occurred after a median period of 12.5 months (range, 0{-}172 months): 1 patient in chronic phase, 2 in accelerated phase, and 7 in blast phase. Eight of ten patients died after a median follow{-}up of 16.5 months (range, 8{-}186 months) following the initial diagnosis of CML, and a median of 6.7 months (range, 0.8{-}16.6 months) after the emergence of 11q23 rearrangement. The remaining two patients had complete remission at the last follow{-}up, 50.2 and 6.9 months, respectively. In addition, we also identified a case with 11q23/t(11;17) in Ph{-}negative cells in a patient with a history of CML. MLL involvement was tested by fluorescence in situ hybridization in 10 cases, and 7 cases (70\%) were positive. CONCLUSIONS In summary, chromosomal rearrangements involving 11q23 are rare in CML, frequently occurring in blast phase, and are often associated with other cytogenetic abnormalities. These patients had a low response rate to tyrosine kinase inhibitors and a poor prognosis. TEXT: Background BCR{-}ABL1 derived... [Truncated] \\ \\ \begin{tabular}{lll} \hline & serious & patientsex \\ \hline target & 1 & 1 \\ flan-t5-large & 1 & 1 \\ gpt-4 & 1 & 1 \\ \hline & \multicolumn{2}{l}{drugs} \\ \hline target & \multicolumn{2}{p{13.2cm}}{bosutinib, dasatinib, imatinib, nilotinib} \\ flan-t5-large & \multicolumn{2}{p{13.2cm}}{hydroxyurea, imatinib} \\ gpt-4 & \multicolumn{2}{p{13.2cm}}{imatinib, dasatinib, nilotinib} \\ \hline & \multicolumn{2}{l}{reactions} \\ \hline target & \multicolumn{2}{p{13.2cm}}{blast cell count increased, chronic myeloid leukaemia transformation, drug ineffective, thrombocytopenia} \\ flan-t5-large & \multicolumn{2}{p{13.2cm}}{blast stage leukaemia} \\ gpt-4 & \multicolumn{2}{p{13.2cm}}{clonal evolution, disease progression, poor prognosis} \\ \hline \end{tabular} \\ \\
 (PMID: 29170802) TITLE: Pharmacokinetics and safety of panitumumab in a patient with chronic kidney disease. ABSTRACT: Data on panitumumab dosing in cancer patients with renal insufficiency are lacking. Here, we report a 63{-}year{-}old metastatic colorectal cancer patient with chronic kidney injury with a glomerular filtration rate of approximately 11~mL/min. Pharmacokinetic parameters, including dose{-}normalized area under the curve, clearance and elimination half{-}life (T 1/2) after the 11th and 12th infusions were estimated using trapezoidal non{-}compartmental methods. Data were compared to previous reported pharmacokinetic data from studies in patients with normal renal function. The results show that the pharmacokinetic data in this patient with kidney failure are comparable to those in patients with adequate renal function. Moreover the treatment was well tolerated in this patient. This study suggests that panitumumab can be safely used in cancer patients with renal impairment without dose adjustment. TEXT: Introduction Panitumumab is a fully humane monoclonal antibody targeting the epidermal growth factor receptor (EGFR) and is registered for the treatment of RAS wild{-}type metastatic colorectal cancer, either alone or combined with chemotherapy. As previously discussed elsewhere, clearance of panitumumab mainly occurs by an EGFR sink. In case of saturation of all receptors, panitumumab will be cleared by immunologic mechanisms, such as complement{-}dependent cytotoxicity (CDC), antibody dependent cell{-}mediated cytotoxicity and apoptosis {[}1{]}. Therefore, theoretically renal insufficiency is not likely to influence the pharmacokinetics of panitumumab. The study of councilman et al. showed that nephrotic syndrome was associated with increased rituximab clearance, and therefore, decreased half{-}life. An possible explanation for the observed effect is loss of monoclonal antibody in the urine and not altered clearance {[}2{]}. The most recent summary of product characteristics (SmPc) of panitumumab states that a population pharmacokinetic analysis (among race, age, gender, hepatic function, concomitant chemotherapy and EGFR membrane{-}staining intensity in tumor cells) renal function does not influence the pharmacokinetics of panitumumab, however, it is not tested in patients. The only available clinical information concerns a case report showing safety and efficacy of panitumumab (combined with oxaliplatin, folic acid and 5{-}FU) in a hemodialysis patient ... [Truncated] \\ \\ \begin{tabular}{lll} \hline & serious & patientsex \\ \hline target & 1 & 1 \\ flan-t5-large & 1 & 1 \\ gpt-4 & 1 & 1 \\ \hline & \multicolumn{2}{l}{drugs} \\ \hline target & \multicolumn{2}{p{13.2cm}}{fluorouracil, folic acid, oxaliplatin} \\ flan-t5-large & \multicolumn{2}{p{13.2cm}}{fluorouracil, leucovorin, oxaliplatin, panitumumab} \\ gpt-4 & \multicolumn{2}{p{13.2cm}}{panitumumab} \\ \hline & \multicolumn{2}{l}{reactions} \\ \hline target & \multicolumn{2}{p{13.2cm}}{product use in unapproved indication, renal impairment} \\ flan-t5-large & \multicolumn{2}{p{13.2cm}}{electrolyte imbalance, skin toxicity} \\ gpt-4 & \multicolumn{2}{p{13.2cm}}{skin toxicity} \\ \hline \end{tabular} \\ \\
 (PMID: 24860718) TITLE: Cardiac safety results from a phase II, open{-}label, multicenter, pilot study of two docetaxel{-}based regimens plus bevacizumab for the adjuvant treatment of subjects with node{-}positive or high{-}risk node{-}negative breast cancer. ABSTRACT: OBJECTIVE Adding antiangiogenic therapy to standard chemotherapy has improved response rates and progression{-}free survival in metastatic breast cancer (BC) patients. This phase II study evaluated cardiac safety of bevacizumab with/without trastuzumab with two docetaxel{-}based regimens in early BC. METHODS 127 women with non{-}metastatic node{-}positive or high{-}risk node{-}negative BC were enrolled. Women with human epidermal growth factor receptor 2 (HER2){-}negative BC (n = 93) received docetaxel/doxorubicin/cyclophosphamide (TAC) + bevacizumab, while women with HER2{-}positive disease (n = 34) received docetaxel/carboplatin/trastuzumab (TCH) + bevacizumab, every 3~weeks for six cycles. Maintenance therapy with bevacizumab alone or bevacizumab plus trastuzumab, respectively, was given every 3~weeks for 52~weeks. The primary objective was to evaluate cardiac safety, as measured by the incidence of ≥ grade 3 clinical congestive heart failure (CHF); the secondary objective was assessment of safety and toxicity. RESULTS At least one cardiac adverse event (AE; CHF, cardiomyopathy, or left ventricular dysfunction) was reported in 26.1\% of TAC (n = 92) and 17.6\% of TCH subjects (n = 34); there were no cardiac deaths. ≥ Grade 3 clinical CHF was observed in 4.3\% in the TAC plus bevacizumab stratum and 0\% in the TCH plus bevacizumab stratum. A ≥ grade 3 treatment{-}emergent AE (any kind) related to study treatment was observed in 59.8\% in the TAC with bevacizumab and 52.9\% in the TCH plus bevacizumab stratum. CONCLUSIONS Adding bevacizumab to a docetaxel{-}based regimen with trastuzumab did not appear to increase cardiotoxicity. BACKGROUND ClinicalTrials.gov Identifier: NCT00446030, registered March 8, 2007. TEXT: Introduction Breast cancer mortality has declined over the past 2 decades; however, it still remains the most common type of cancer in women, accounting for an estimated 29\% of all new cases (Siegel et al. 2014). The 5{-}year survival rate for women with breast cancer is 99\% for those with localized disease and 84\% for regional disease, and only 24\% in patients with distant disease (Siegel et al. 2014). Several studies in human epidermal growth factor receptor 2 (HER2){-}normal metastatic... [Truncated] \\ \\ \begin{tabular}{lll} \hline & serious & patientsex \\ \hline target & 1 & 2 \\ flan-t5-large & 1 & 2 \\ gpt-4 & 1 & 2 \\ \hline & \multicolumn{2}{l}{drugs} \\ \hline target & \multicolumn{2}{p{13.2cm}}{bevacizumab, cyclophosphamide, docetaxel, doxorubicin hydrochloride} \\ flan-t5-large & \multicolumn{2}{p{13.2cm}}{bevacizumab, carboplatin, docetaxel, doxorubicin, trastuzumab} \\ gpt-4 & \multicolumn{2}{p{13.2cm}}{bevacizumab, trastuzumab, docetaxel, doxorubicin, cyclophosphamide, carboplatin} \\ \hline & \multicolumn{2}{l}{reactions} \\ \hline target & \multicolumn{2}{p{13.2cm}}{clostridial infection} \\ flan-t5-large & \multicolumn{2}{p{13.2cm}}{cardiac failure congestive} \\ gpt-4 & \multicolumn{2}{p{13.2cm}}{congestive heart failure, cardiomyopathy, left ventricular dysfunction} \\ \hline \end{tabular} \\\end{longtable}
